\documentclass[letterpaper]{exam}
\printanswers
\usepackage{hyperref}
\usepackage[utf8x]{inputenc}
\usepackage[table]{xcolor}
\usepackage{listings}
\usepackage{mdframed}
\usepackage{lmodern}
\usepackage[left=0.25in, right=0.25in, top=0.75in, bottom=0.75in]{geometry}
\usepackage{graphicx}
\usepackage{amsmath,amssymb}
\usepackage{tikz}
\usepackage{pgfplots}
\usepackage{subfigure}
\usepackage{enumerate}
\usepackage{tcolorbox}
\usepackage{float}
\pagestyle{headandfoot}

\usepackage{cancel}
\usepackage{placeins}
\usepackage{multirow}
\usepackage{algorithm2e}
\pagecolor{black}
\color{white}
\hypersetup{%
  colorlinks=true,% hyperlinks will be black
  linkbordercolor=red,% hyperlink borders will be red
  pdfborderstyle={/S/U/W 1}% border style will be underline of width 1pt
}

\newcommand{\soln}{\\ \textbf{Solution}: }
\newcommand{\bkt}[1]{\left(#1\right)}
\lhead{MA5892: Numerical Methods and Scientific Computing}

\chead{Assignment: 3}

\rhead{Rollnumber: EE18B067}

\begin{document}
\hrule
\vspace{3mm}
\noindent 
\vspace{3mm}
\noindent{{\sf Roll No:} EE18B067 \hfill  {\sf Name:Abhishek Sekar}}% put your ROLL NO AND NAME HERE

\noindent
{{\sf Date: \today }} %Date



\vspace{3mm}
\hrule
\begin{questions}
\question{\sc [Pade Approximation]}\\
Numerically solve the following boundary value problem with Neumann boundary conditions:\\
\begin{align}
    \frac{d^2y}{dx^2} + y &= x^3, &\mbox{ 0 $\leq$ x $\leq$ 1} 
\end{align}
with $y^{'}$(0) = $y^{'}$(1) = 0.We discretize the domain using grid points $x_{i}$ = (i-0.5)h, $i \in \{1,2,\ldots,N\}$. Note that there are no grid points on the boundaries. In this problem, $y_{i}$ is the numerical estimate of y at node $x_{i}$. By using a finite difference scheme, we can estimate $y_{i}^{"}$ in terms of linear combinations of $y_{i}$'s and transform the ODE into a linear system of equations.
\begin{parts}
\part Derive a fourth order Pade approximation for the second derivative at the $i^{th}$ node involving only its neighbours i $\pm$ 1,i.e., obtain $y_{i}^{"}$ involving $y_{i \pm 1},y_{i},y_{i \pm 1}^{"}$.Note that this is applicable only at i $\in \{2,3,\ldots,N-1\}$
\begin{solution}
\\
\textbf{Pade Approximation:}\\
We want to express $y_i^{"}$ as a linear combination of $y_{i \pm 1},y_{i},y_{i \pm 1}^{"}$.Therefore,we have,
\begin{align}\label{1}
    y_i^{"} = a_0y_i + a_1y_{i+1} + a_2y_{i-1} + a_3y_{i+1}^{"} + a_4y_{i-1}^{"} + \mathcal{O}(h^4)
\end{align}
Where, $a_0,a_1,a_2,a_3,a_4$ are constants.\\
To determine these constants we can make use of the taylor table.\\
Expanding the terms,$y_{i \pm 1},y_{i},y_{i \pm 1}^{"}$ using the taylor series about the $i^{th}$ point where y is $y_i$ and making use of the fact that two successive datapoints are separated by h, we can write the below taylor table.\\
\begin{center}
\textbf{Taylor Table}\\
\renewcommand{\arraystretch}{2}
\begin{tabular}{|c|c|c|c|c|c|c|c|}
\hline

&$y_i$&$y_i^{(i)}$&$y_i^{(ii)}$&$y_i^{(iii)}$&$y_i^{(iv)}$&$y_i^{(v)}$&$y_i^{(vi)}$\\
\hline

     $y_i^{"}$&0&0&1&0&0&0&0\\
\hline

     $a_0y_i$&$a_0$&0&0&0&0&0&0\\
\hline

     $a_1y_{i+1}$&$a_1$&$a_1h$&$a_1\frac{h^2}{2}$&$a_1\frac{h^3}{6}$&$a_1\frac{h^4}{24}$&$a_1\frac{h^5}{120}$&$a_1\frac{h^6}{720}$\\
\hline
     $a_2y_{i-1}$&$a_2$&$-a_2h$&$a_2\frac{h^2}{2}$&$-a_2\frac{h^3}{6}$&$a_2\frac{h^4}{24}$&$-a_2\frac{h^5}{120}$&$a_2\frac{h^6}{720}$\\
\hline
     $a_3y_{i+1}^{"}$&0&0&$a_3$&$a_3h$&$a_3\frac{h^2}{2}$&$a_3\frac{h^3}{6}$&$a_3\frac{h^4}{24}$\\
\hline
     $a_4y_{i-1}^{"}$&0&0&$a_4$&$-a_4h$&$a_4\frac{h^2}{2}$&$-a_4\frac{h^3}{6}$&$a_4\frac{h^4}{24}$\\
\hline
\end{tabular}
\end{center}
Using the first five columns of the taylor table we can now find the constants $a_0,a_1,a_2,a_3,a_4$ by observing equation \ref{1} and comparing coefficients of similar terms.\\
Therefore, from this, we have,
\begin{align*}
    &a_0 + a_1 + a_2 = 0 &\mbox{ (From column 1)}\\
    &a_1h - a_2h     = 0 &\mbox{ (From column 2)}\\
    &a_1\frac{h^2}{2} + a_2\frac{h^2}{2} + a_3 + a_4 = 1 &\mbox{ (From column 3)}\\
    &a_1\frac{h^3}{6} - a_2\frac{h^3}{6} + a_3h - a_4h = 0 &\mbox{ (From column 4)}\\
    &a_1\frac{h^4}{24} + a_2\frac{h^4}{24} + a_3\frac{h^2}{2} + a_4\frac{h^2}{2} = 0 &\mbox{ (From column 5)}
\end{align*}
Looking at the second and the fourth equations, we see that, $a_1 = a_2$ and $a_3 = a_4$. Therefore, from the first equation, we have, $a_0 = -2a_1$ and from the fifth equation,we have, $a_3 = -a_1 \frac{h^2}{12}$. Using this information and substituting this in the third equation gives us,
\begin{align*}
    &a_1\left(2\cdot\frac{h^2}{2} -2\cdot\frac{h^2}{12}\right) = 1\\
    &a_1 = \frac{6}{5h^2}
\end{align*}
From this, we get the values of the constants to be,
\begin{itemize}
    \item $a_0 = -\frac{12}{5h^2}$
    \item $a_1 = \frac{6}{5h^2}$
    \item $a_2 = \frac{6}{5h^2}$
    \item $a_3 = -\frac{1}{10}$
    \item $a_4 = -\frac{1}{10}$ 
\end{itemize}
Therefore, our fourth order Pade approximation scheme is as follows,
\begin{align}\label{3}
    y_i^{"} + \frac{1}{10}y_{i+1}^{"} + \frac{1}{10}y_{i-1}^{"} = -\frac{12}{5h^2}y_i + \frac{6}{5h^2}y_{i+1} + \frac{6}{5h^2}y_{i-1} + \mathcal{O}(h^4)
\end{align}
We can see that this approximation scheme is indeed fourth order by evaluating this approximation along the other columns of the taylor table. Doing this along the sixth column results in a zero, as $a_1 = a_2$ and $a_3 = a_4$. Therefore, the highest order term remaining has an order $\mathcal{O}(h^4)$.
\end{solution}
\part For the left boundary, derive a third order Pade scheme to approximate $y_{1}^{"}$ in the following form:\\
\begin{align*}
    y_{1}^{"}+b_{2}y_{2}^{"} = a_{1}y_{1} + a_{2}y_{2} + a_{3}y_{3} + a_{4}y^{'}(0) + \mathcal{O}\left(h^3\right)
\end{align*}
\begin{solution}
\\
\textbf{Pade Approximation for left boundary}\\
We can proceed as we did in the last subdivision by implementing the taylor table by expanding the taylor series of all terms about the first point,where y is $y_1$. We see that $y_1$ is obtained at the point $x_1 = 0.5h$ and can use this fact to expand $y^{'}(0)$ about $y_1$.\\
Using the constants given in the question, we can write the Taylor table as follows,
\begin{center}
\textbf{Taylor Table}\\
\renewcommand{\arraystretch}{2}
\begin{tabular}{|c|c|c|c|c|c|c|}
\hline

&$y_1$&$y_1^{(i)}$&$y_1^{(ii)}$&$y_1^{(iii)}$&$y_1^{(iv)}$&$y_1^{(v)}$\\
\hline

     $y_1^{"}$&0&0&1&0&0&0\\
\hline

     $a_1y_1$&$a_1$&0&0&0&0&0\\
\hline

     $a_2y_{2}$&$a_2$&$a_2h$&$a_2\frac{h^2}{2}$&$a_2\frac{h^3}{6}$&$a_2\frac{h^4}{24}$&$a_2\frac{h^5}{120}$\\
\hline
     $a_3y_{3}$&$a_3$&$a_3\cdot2h$&$a_3\cdot\frac{4h^2}{2}$&$a_3\cdot\frac{8h^3}{6}$&$a_3\cdot\frac{16h^4}{24}$&$a_3\cdot\frac{32h^5}{120}$\\
\hline
     $a_4y^{'}(0)$&0&$a_4$&$-a_4\frac{h}{2}$&$a_4\frac{h^2}{8}$&$-a_4\frac{h^3}{48}$&$a_4\frac{h^4}{384}$\\
\hline
     $b_2y_{2}^{"}$&0&0&$b_2$&$b_2h$&$b_2\frac{h^2}{2}$&$b_2\frac{h^3}{6}$\\
\hline
\end{tabular}
\end{center}
Like before, we try finding the constants by using the first 5 columns of the Taylor table.\\
This yields us with the following five equations,
\begin{align*}
    &a_1 + a_2 + a_3 = 0 &\mbox{ (From column 1) }\\
    &a_2h + 2a_3h + a_4 = 0 &\mbox{ (From column 2)}\\
    &a_2\frac{h^2}{2} + 2a_3h^2 - a_4\frac{h}{2}  = 1 +b_2 &\mbox{ (From column 3)}\\
    &a_2\frac{h^3}{6} + a_3\frac{4h^3}{3} + a_4\frac{h^2}{8} = b_2h &\mbox{ (From column 4)}\\
    &a_2\frac{h^4}{24} + a_3\frac{2h^4}{3} - a_4\frac{h^3}{48}= b_2\frac{h^2}{2} &\mbox{ (From column 5)}
\end{align*}
These equations get really messy to solve as they possess several powers of h alongside constants, and therefore, we implement a small trick.\\
Let $\acute{a_1},\acute{a_2},\acute{a_3}$ denote $a_1,a_2,a_3$ multiplied by $h^2$, i.e., $\frac{\acute{a_1}}{h^2} = a_1$. Similarly, let $\acute{a_4} = a_4h$.\\
Then the above equations simplify to,
\begin{align*}
    &\acute{a_1} + \acute{a_2} + \acute{a_3} = 0 &\mbox{ (From column 1) }\\
    &\acute{a_2} + 2\acute{a_3} + \acute{a_4} = 0 &\mbox{ (From column 2)}\\
    &\frac{\acute{a_2}}{2} + \acute{2a_3} - \frac{\acute{a_4}}{2}  = 1 +b_2 &\mbox{ (From column 3)}\\
    &\frac{\acute{a_2}}{6} + \frac{4\acute{a_3}}{3} + \frac{\acute{a_4}}{8} = b_2 &\mbox{ (From column 4)}\\
    &\frac{\acute{a_2}}{24} + \frac{2\acute{a_3}}{3} - \frac{\acute{a_4}}{48}= \frac{b_2}{2} &\mbox{ (From column 5)}
\end{align*}
Now,on implementing this trick, all the terms involving h are cancelled, therefore making this easy to compute using Gaussian elimination. Writing the above set of equations in matrix form(by bringing $b_2$ to the L.H.S), we have,
\begin{align*}
\renewcommand{\arraystretch}{2}
    \begin{bmatrix}
    1 & 1 & 1&0&0\\
    0&1&2&1&0\\
    0&\frac{1}{2}&2&-\frac{1}{2}&-1\\
    0&\frac{1}{6}&\frac{4}{3}&-\frac{1}{8}&-1\\
    0&\frac{1}{24}&\frac{2}{3}&-\frac{1}{48}&-\frac{1}{2}\\
    \end{bmatrix} \cdot \begin{bmatrix}
    \acute{a_1}\\
    \acute{a_2}\\
    \acute{a_3}\\
    \acute{a_4}\\
    b_2\\ 
    \end{bmatrix}
     = 
     \begin{bmatrix}
     0\\
     0\\
     1\\
     0\\
     0\\
     \end{bmatrix}
\end{align*}
Skipping the steps of the Gaussian elimination process, we end up with the following values for the constants,
\begin{itemize}
    \item $\acute{a_1} = -\frac{36}{23}$
    \item $\acute{a_2} = \frac{48}{23}$
    \item $\acute{a_3} = -\frac{12}{23}$
    \item $\acute{a_4} = -\frac{24}{23}$
    \item $b_2 = -\frac{11}{23}$
\end{itemize}
From this, we can write the third order Pade scheme at the left boundary in the following manner:
\begin{align}\label{4}
    y_{1}^{"}+ -\frac{11}{23}y_{2}^{"} = \frac{1}{h^2}\left(-\frac{36}{23}y_{1} + \frac{48}{23}y_{2} + -\frac{12}{23}y_{3}\right) + -\frac{24}{23h}y^{'}(0) + \mathcal{O}\left(h^3\right)
\end{align}
\end{solution}
\newpage
\part Repeat the above for the right boundary.
\begin{solution}
\\
\textbf{Pade Approximation for right boundary}\\
As in the left boundary, we can devise a third order Pade approximation in the right boundary as well.
Therefore, we'll have,
\begin{align*}
    y_{N}^{"}+b_{N-1}y_{N-1}^{"} = a_{N}y_{N} + a_{N-1}y_{N-1} + a_{N-2}y_{N-2} + a_{N+1}y^{'}(1) + \mathcal{O}\left(h^3\right)
\end{align*}
Where, $b_{N-1},a_N,a_{N-1},a_{N-2},a_{N+1}$ are constants.\\
As we did before, we can try using a Taylor table to find these constants by expanding every term as a taylor series about the $N^{th}$ point where y takes the value $y_{N}$. As in the previous case, we observe that $x_{N}$ is just 0.5h away from 1 as $h = \frac{1}{N}$.\\
Therefore, using the above constants, we have,
\begin{center}
\textbf{Taylor Table}\\
\renewcommand{\arraystretch}{2}
\begin{tabular}{|c|c|c|c|c|c|c|}
\hline

&$y_N$&$y_N^{(i)}$&$y_N^{(ii)}$&$y_N^{(iii)}$&$y_N^{(iv)}$&$y_N^{(v)}$\\
\hline

     $y_N^{"}$&0&0&1&0&0&0\\
\hline

     $a_N y_N$&$a_N$&0&0&0&0&0\\
\hline

     $a_{N-1}y_{N-1}$&$a_{N-1}$&$-a_{N-1}h$&$a_{N-1}\frac{h^2}{2}$&$-a_{N-1}\frac{h^3}{6}$&$a_{N-1}\frac{h^4}{24}$&$-a_{N-1}\frac{h^5}{120}$\\
\hline
     $a_{N-2}y_{N-2}$&$a_{N-2}$&$-a_{N-2}\cdot2h$&$a_{N-2}\cdot\frac{4h^2}{2}$&$-a_{N-2}\cdot\frac{8h^3}{6}$&$a_{N-2}\cdot\frac{16h^4}{24}$&$-a_{N-2}\cdot\frac{32h^5}{120}$\\
\hline
     $a_{N+1}y^{'}(1)$&0&$a_{N+1}$&$a_{N+1}\frac{h}{2}$&$a_{N+1}\frac{h^2}{8}$&$a_{N+1}\frac{h^3}{48}$&$a_{N+1}\frac{h^4}{384}$\\
\hline
     $b_{N-1}y_{2}^{"}$&0&0&$b_{N-1}$&$-b_{N-1}h$&$b_{N-1}\frac{h^2}{2}$&$b_{N-1}\frac{h^3}{6}$\\
\hline
\end{tabular}
\end{center}
On looking at the above Taylor table, we see that it is very similar to what we saw in the Taylor table at the left boundary, but with terms involving h replaced by those involving -h. This becomes clear if we write down the equations to find the coefficients by considering the first five columns of the Taylor Table.\\
\begin{align*}
    &a_{N} + a_{N-1} + a_{N-2} = 0 &\mbox{ (From column 1) }\\
    &-a_{N-1}h + -2a_{N-2}h + a_{N+1} = 0 &\mbox{ (From column 2)}\\
    &a_{N-1}\frac{h^2}{2} + 2a_{N-2}h^2 - a_{N+1}\frac{h}{2}  = 1 +b_{N-1} &\mbox{ (From column 3)}\\
    &-a_{N-1}\frac{h^3}{6} - a_{N-2}\frac{4h^3}{3} + a_{N+1}\frac{h^2}{8} = -b_{N-1}h &\mbox{ (From column 4)}\\
    &a_{N-1}\frac{h^4}{24} + a_{N-2}\frac{2h^4}{3} + a_{N+1}\frac{h^3}{48}= b_{N-1}\frac{h^2}{2} &\mbox{ (From column 5)}
\end{align*}
\\
If we implement a similar simplication scheme as we did last time for the left boundary but with a slight modification, i.e., replacing h with -h in the scheme, we have,
\begin{itemize}
    \item $\acute{a_{N}} = h^2a_{N}$
    \item $\acute{a_{N-1}} = h^2a_{N-1}$
    \item $\acute{a_{N-2}} = h^2a_{N-2}$
    \item $\acute{a_{N+1}} = -ha_{N+1}$
\end{itemize}
On using these in the above equations, we get,
\begin{align*}
    &\acute{a_{N}} + \acute{a_{N-1}} + \acute{a_{N-2}} = 0 &\mbox{ (From column 1) }\\
    &\acute{a_{N-1}} + 2\acute{a_{N-2}} + \acute{a_{N+1}} = 0 &\mbox{ (From column 2)}\\
    &\frac{\acute{a_{N-1}}}{2} + \acute{2a_{N-2}} - \frac{\acute{a_{N+1}}}{2}  = 1 +b_{N-1} &\mbox{ (From column 3)}\\
    &\frac{\acute{a_{N-1}}}{6} + \frac{4\acute{a_{N-2}}}{3} + \frac{\acute{a_{N+1}}}{8} = b_{N-1} &\mbox{ (From column 4)}\\
    &\frac{\acute{a_{N-1}}}{24} + \frac{2\acute{a_{N-2}}}{3} - \frac{\acute{a_{N+1}}}{48}= \frac{b_{N-1}}{2} &\mbox{ (From column 5)}
\end{align*}
These are the exact same equations we got at the left boundary with different constant labels.\\
Therefore, comparing the constants with those at the left boundary, we can write the third order Pade scheme at the right boundary as follows:
\begin{align}\label{5}
   y_{N}^{"}+ -\frac{11}{23}y_{N-1}^{"} = \frac{1}{h^2}\left(-\frac{36}{23}y_{N} + \frac{48}{23}y_{N-1} + -\frac{12}{23}y_{N-2}\right) + \frac{24}{23h}y^{'}(1) + \mathcal{O}\left(h^3\right) 
\end{align}
\end{solution}
\part Use the finite difference formulae derived above to obtain a linear system for $y_{i}^{"}$. Explicitly write down the entries in the matrix and the right hand side.
\begin{solution}
\\
\textbf{The overall Pade scheme}\\
From \ref{3},\ref{4},\ref{5}, we have the following equations,
\begin{align*}
  & y_i^{"} + \frac{1}{10}y_{i+1}^{"} + \frac{1}{10}y_{i-1}^{"} = -\frac{12}{5h^2}y_i + \frac{6}{5h^2}y_{i+1} + \frac{6}{5h^2}y_{i-1} + \mathcal{O}(h^4)\\
  &y_{1}^{"}+ -\frac{11}{23}y_{2}^{"} = \frac{1}{h^2}\left(-\frac{36}{23}y_{1} + \frac{48}{23}y_{2} + -\frac{12}{23}y_{3}\right) + -\frac{24}{23h}y^{'}(0) + \mathcal{O}\left(h^3\right) &\mbox{ (Left Boundary)}\\
  &y_{N}^{"}+ -\frac{11}{23}y_{N-1}^{"} = \frac{1}{h^2}\left(-\frac{36}{23}y_{N} + \frac{48}{23}y_{N-1} + -\frac{12}{23}y_{N-2}\right) + \frac{24}{23h}y^{'}(1) + \mathcal{O}\left(h^3\right) &\mbox{ (Right Boundary)}\\
\end{align*}
We can generalize \ref{3} for all i $\in \{2,3,4,\ldots,N-1\}$\\
Therefore, doing that, we have,
\begin{align*}
  &y_{1}^{"}+ -\frac{11}{23}y_{2}^{"} = \frac{1}{h^2}\left(-\frac{36}{23}y_{1} + \frac{48}{23}y_{2} + -\frac{12}{23}y_{3}\right) + -\frac{24}{23h}y^{'}(0) + \mathcal{O}\left(h^3\right) &\mbox{ (Left Boundary)}\\
  & y_2^{"} + \frac{1}{10}y_{3}^{"} + \frac{1}{10}y_{1}^{"} = -\frac{12}{5h^2}y_2 + \frac{6}{5h^2}y_{3} + \frac{6}{5h^2}y_{1} + \mathcal{O}(h^4)\\
  & y_3^{"} + \frac{1}{10}y_{4}^{"} + \frac{1}{10}y_{2}^{"} = -\frac{12}{5h^2}y_3 + \frac{6}{5h^2}y_{4} + \frac{6}{5h^2}y_{2} + \mathcal{O}(h^4)\\
  &\vdots\\
  & y_{N-1}^{"} + \frac{1}{10}y_{N}^{"} + \frac{1}{10}y_{N-2}^{"} = -\frac{12}{5h^2}y_{N-1} + \frac{6}{5h^2}y_{N} + \frac{6}{5h^2}y_{N-2} + \mathcal{O}(h^4)\\
  &y_{N}^{"}+ -\frac{11}{23}y_{N-1}^{"} = \frac{1}{h^2}\left(-\frac{36}{23}y_{N} + \frac{48}{23}y_{N-1} + -\frac{12}{23}y_{N-2}\right) + \frac{24}{23h}y^{'}(1) + \mathcal{O}\left(h^3\right) &\mbox{ (Right Boundary)}\\
\end{align*}
Ignoring the residual error terms,we can write the above set of equations in matrix form as follows\\
\begin{align*}
    \renewcommand{\arraystretch}{2}
    \begin{bmatrix}
    1&-\frac{11}{23}&0&\ldots&\ldots&0\\
    \frac{1}{10}&1&\frac{1}{10}&\ldots&\ldots&0\\
    \vdots&\ddots&\ddots&\ddots&\ddots&0\\
    0&\ldots&\ldots&\frac{1}{10}&1&\frac{1}{10}\\
    0&\ldots&\ldots&0&-\frac{11}{23}&1
    \end{bmatrix}_{N\text{x}N}\cdot \begin{bmatrix}
    y^{"}_1\\
    y^{"}_2\\
    \vdots\\
    y^{"}_{N-1}\\
    y^{"}_{N}\\
    \end{bmatrix}_{N\text{x}1} = \frac{1}{h^2}\cdot \begin{bmatrix}
    \left(-\frac{36}{23}y_{1} + \frac{48}{23}y_{2} + -\frac{12}{23}y_{3}\right) - \frac{24h}{23}y^{'}(0)\\
    -\frac{12}{5}y_2 + \frac{6}{5}y_{3} + \frac{6}{5}y_{1}\\
    \vdots\\
    -\frac{12}{5}y_{N-1} + \frac{6}{5}y_{N} + \frac{6}{5}y_{N-2}\\
    \left(-\frac{36}{23}y_{N} + \frac{48}{23}y_{N-1} + -\frac{12}{23}y_{N-2}\right) + \frac{24h}{23}y^{'}(1)\\
    \end{bmatrix}_{N\text{x}1}
\end{align*}
If we enforce the Neumann boundary conditions in the above equations,then,
\begin{itemize}
    \item $y^{'}(0) = 0$
    \item $y^{'}(1) = 0$
\end{itemize}
Given this, we can write the vector on the R.H.S also as a matrix vector product as follows,
\begin{align}\label{6}
  \renewcommand{\arraystretch}{2}
    \begin{bmatrix}
    1&-\frac{11}{23}&0&\ldots&\ldots&0\\
    \frac{1}{10}&1&\frac{1}{10}&\ldots&\ldots&0\\
    \vdots&\ddots&\ddots&\ddots&\ddots&0\\
    0&\ldots&\ldots&\frac{1}{10}&1&\frac{1}{10}\\
    0&\ldots&\ldots&0&-\frac{11}{23}&1
    \end{bmatrix}_{N\text{x}N}\cdot \begin{bmatrix}
    y^{"}_1\\
    y^{"}_2\\
    \vdots\\
    y^{"}_{N-1}\\
    y^{"}_{N}\\
    \end{bmatrix}_{N\text{x}1} = \frac{1}{h^2}\begin{bmatrix}
    -\frac{36}{23}&\frac{48}{23}&-\frac{12}{23}&\ldots&\ldots&0\\
    \frac{6}{5}&-\frac{12}{5}&\frac{6}{5}&\ldots&\ldots&0\\
    \vdots&\ddots&\ddots&\ddots&\ddots&0\\
    0&\ldots&\ldots&\frac{6}{5}&-\frac{12}{5}&\frac{6}{5}\\
    0&\ldots&\ldots&-\frac{12}{23}&\frac{48}{23}&-\frac{36}{23}
    \end{bmatrix}_{N\text{x}N}\cdot \begin{bmatrix}
    y_1\\
    y_2\\
    \vdots\\
    y_{N-1}\\
    y_{N}\\
    \end{bmatrix}_{N\text{x}1} 
\end{align}
Let us write the above equation in a compact notation as follows,
\begin{align*}
    &A\vec{y^{"}} = B\vec{y}\\
    &A =  \renewcommand{\arraystretch}{2}\begin{bmatrix}
    1&-\frac{11}{23}&0&\ldots&\ldots&0\\
    \frac{1}{10}&1&\frac{1}{10}&\ldots&\ldots&0\\
    \vdots&\ddots&\ddots&\ddots&\ddots&0\\
    0&\ldots&\ldots&\frac{1}{10}&1&\frac{1}{10}\\
    0&\ldots&\ldots&0&-\frac{11}{23}&1
    \end{bmatrix}\\
    &B = \renewcommand{\arraystretch}{2}\frac{1}{h^2}\begin{bmatrix}
    -\frac{36}{23}&\frac{48}{23}&-\frac{12}{23}&\ldots&\ldots&0\\
    \frac{6}{5}&-\frac{12}{5}&\frac{6}{5}&\ldots&\ldots&0\\
    \vdots&\ddots&\ddots&\ddots&\ddots&0\\
    0&\ldots&\ldots&\frac{6}{5}&-\frac{12}{5}&\frac{6}{5}\\
    0&\ldots&\ldots&-\frac{12}{23}&\frac{48}{23}&-\frac{36}{23}
    \end{bmatrix}\\
    &\vec{y^{"}} = \renewcommand{\arraystretch}{2}\begin{bmatrix}
    y^{"}_1\\
    y^{"}_2\\
    \vdots\\
    y^{"}_{N-1}\\
    y^{"}_{N}\\
    \end{bmatrix}
    &\vec{y} = \renewcommand{\arraystretch}{2}\begin{bmatrix}
    y_1\\
    y_2\\
    \vdots\\
    y_{N-1}\\
    y_{N}\\
    \end{bmatrix}
\end{align*}
Let us denote the matrix of the cubes of the nodes,$\vec{x^3} = \begin{bmatrix}
x^3_1\\
    x^3_2\\
    \vdots\\
    x^{3}_{N-1}\\
    x^{3}_{N}\\
\end{bmatrix}$
,then by using the differential equation given to us, we can write $\vec{y^{"}} = \vec{x^3} - \vec{y}$ and similarly $\vec{y} = \vec{x^3} - \vec{y^{"}}$ \\
Using these in our matrix equation, we get,
\begin{align}\label{7}
    &(A+B)\vec{y} = A\vec{x^3}\\
    &(A+B)\vec{y^{"}} = B\vec{x^3}
\end{align}
The above system of linear equations can be easily solved numerically as A is a tridiagonal system and B is a four diagonal system.
\end{solution}
\part Compare the numerical and exact solution by varying n $\in \{10,20,50,100,200,500,1000\}$. Plot the error (computed using the max-norm as a function of n) on a log-log plot. Discuss your result.
\begin{solution}
\\
\textbf{Exact solution}\\
First let us compute the exact solution of the given differential equation.\\
There are two parts to the our solution y(x),i.e.,
\begin{align*}
    y(x) = y_p(x) + y_h(x)
\end{align*}
where, $y_p(x)$ and $y_h(x)$ are the particular and homogenous solutions to the given differential equation.\\
On inspection, we can observe that the particular solution, $y_p(x) = x^3 - 6x$.\\ The homogenous solution to this differential equation is of the form $y_h(x) = c_1\sin{x}+c_2\cos{x}$ where $c_1,c_2$ are constants. We can enforce the Neumann boundary conditions to find the constants $c_1$ and $c_2$.\\
Doing that, we have,
\begin{align*}
    y^{'}(x) = c_1 \cos{x} + c_2 -\sin{x} + 3x^2 - 6\\
    y^{'}(0) = c_1 - 6 = 0 \Rightarrow c_1 = 6\\
    y^{'}(1) = 6 \cos{1} + c_2-\sin{1} + 3-6 = 0 \Rightarrow c_2 = \frac{6\cos{1}-3}{\sin{1}}
\end{align*}
 Therefore,
 \begin{equation}
     y(x) = x^3 - 6x + 6\sin{x} + \left(\frac{6\cos{1}-3}{\sin{1}}\right)\cos{x}
 \end{equation}
The numerical solution can be found using the equation \ref{7} derived in the previous section.
\\
\textbf{Plots}
\begin{figure}[H]  
     \centering
     %\begin{subfigure}[b]{0.3\textwidth}
         %\centering
    \includegraphics[width=12cm]{y.png}
     \label{fig:Dendrogram for the problem 3(c)}
     %\end{subfigure}
     \caption{Plot of the actual function using Pade scheme and analytically}
\end{figure}
\begin{figure}[H]  
     \centering
     %\begin{subfigure}[b]{0.3\textwidth}
         %\centering
    \includegraphics[width=12cm]{y_dd.png}
     \label{fig:Dendrogram for the problem 3(c)}
     %\end{subfigure}
     \caption{Plot of the second derivative using Pade scheme and analytically}
\end{figure}
\begin{figure}[H]  
     \centering
     %\begin{subfigure}[b]{0.3\textwidth}
         %\centering
    \includegraphics[width=12cm]{e_y.png}
     \label{fig:Dendrogram for the problem 3(c)}
     %\end{subfigure}
     \caption{Loglog plot of error in y and its double derivative vs n}
\end{figure}
\begin{figure}[H]  
     \centering
     %\begin{subfigure}[b]{0.3\textwidth}
         %\centering
    \includegraphics[width=12cm]{e_y_e.png}
     \label{fig:Dendrogram for the problem 3(c)}
     %\end{subfigure}
     \caption{Loglog plot of error in y at the endpoints vs n}
\end{figure}
\begin{figure}[H]  
     \centering
     %\begin{subfigure}[b]{0.3\textwidth}
         %\centering
    \includegraphics[width=12cm]{e_y_e_.png}
     \label{fig:Dendrogram for the problem 3(c)}
     %\end{subfigure}
     \caption{Loglog plot of error in the double derivative of y at the endpoints vs n}
\end{figure}
\textbf{Observations}
\begin{itemize}
    \item As expected we see that the numerical method of computing y and $y^{"}$ using the Pade scheme works well in practice as shown by the first two plots. We can not see any difference between the two curves with the naked eye.
    \item When we look at the log log plot of the error in y and the error in $y^{"}$ vs n, we see that the error is of the order of $\mathcal{O}(h^4)$ which is how we designed the Pade scheme. This just reinforces what we'd derived above.
    \item Surprisingly, when we look at the last two plots, we see that the error at the boundary nodes is of the order of $\mathcal{O}(h^4)$ and not $\mathcal{O}(h^3)$ as we'd designed it to be. This is because we know the accurate value of the first derivative at the endpoints thanks to the Neumann boundary conditions, and therefore from the Taylor table, we see that the error is of the order $\mathcal{O}(h^4)$.
\end{itemize}
\end{solution}
\part How are the Neumann boundary conditions enforced into the discretized boundary value problem?
\begin{solution}
\\
Equation \ref{7}, which computes the $\vec{y}$ and $\vec{y^{"}}$ as linear transformations of $\vec{x^3}$ in accordance to the Pade scheme, was computed from \ref{6} which was obtained on enforcing the Neumann boundary conditions. When we computed \ref{4} and \ref{5}, we computed them as third order Pade schemes by including the derivatives at the endpoints as we know them accurately thanks to the Neumann boundary conditions.Only after and not before computing the coefficients, was the boundary conditions enforced, thus preserving the sanctity of the equations derived. This was also visualized in practice by the last two plots of the previous section.
\end{solution}
\end{parts}
\end{questions}
\end{document}